\documentclass[14pt]{extarticle}

\usepackage[T2A]{fontenc}
\usepackage[utf8]{inputenc}
\usepackage[russian]{babel}
\usepackage{hyphenat}

%\usepackage{mathptmx}
\usepackage{gensymb}
\usepackage[left = \flqq{},right = \frqq{}]{dirtytalk}

\usepackage{amsmath}

\usepackage{geometry}
\geometry{
	a4paper,
	top=20mm,
	bottom=23mm,
	footskip=10mm,
	left=25mm,
	right=15mm
}
\setlength{\parindent}{10mm}
\renewcommand{\baselinestretch}{1.5}

\usepackage{fancyhdr}
\pagestyle{fancy}
\renewcommand{\headrulewidth}{0pt}
\lhead{}
\chead{}
\rhead{}
\lfoot{}
\cfoot{}
\rfoot{}

\usepackage{indentfirst}

\usepackage{enumitem}

\usepackage{graphicx}
\usepackage[labelformat=simple]{subcaption}
\renewcommand{\thesubfigure}{\textit{\asbuk{subfigure})}}
\usepackage{setspace}
\usepackage[font=small, labelsep=period]{caption}
\captionsetup[figure]{font={small,stretch=1.0}}

\usepackage{titlesec}
\titleformat{\section}[hang]{\large\bf}{\thesection.}{.5em}{}{}
\titleformat{\subsection}[hang]{\normalsize\bf}{\thesubsection.}{.5em}{}{}
\titleformat{\subsubsection}[hang]{\normalsize\bf}{\thesubsubsection.}{.5em}{}{}
\titlespacing*{\section}{0mm}{6pt}{6pt}
\titlespacing*{\subsection}{0mm}{6pt}{6pt}
\titlespacing*{\subsubsection}{0mm}{6pt}{6pt}

\usepackage{titletoc}
\titlecontents{section}[3mm]{}{\thecontentslabel.\space\filright}{}{\titlerule*[1pc]{.}\contentspage}
\titlecontents{subsection}[6mm]{}{\thecontentslabel.\space\filright}{}{\titlerule*[1pc]{.}\contentspage}
\titlecontents{subsubsection}[9mm]{}{\thecontentslabel.\space\filright}{}{\titlerule*[1pc]{.}\contentspage}

\usepackage{enumitem}
\setlist{nosep}

\begin{document}
	\setlength{\abovedisplayskip}{6pt}
	\setlength{\belowdisplayskip}{6pt}
	\setlength{\belowcaptionskip}{-15pt}
	\thispagestyle{fancy}
\begin{center}
    \textbf{Разделение близких кластеров в калориметре детектора КМД-3}
\end{center}

\noindent\textbf{Баженов Артём Андреевич}

\noindentФизический факультет. Курсовая работа.

\noindentГруппа № 17352, 6 семестр, 2020 год.

\noindentНаучный руководитель:

\noindent\textbf{Семенов Александр Владимирович}

\noindent\textbf{Аннотация}

Данная работа посвящена разработке алгоритма разделения близких кластеров в жидкоксеноновом LXe калориметре детектора КМД-3. Подготовлен для обработки формат данных с калориметра. Реализован и проверен алгоритм классификации тестовых данных на основе метода BDT (boosted decision trees). Доля ложных распознаваний составила 14,3\%. Планируется развить алгоритм для более качественной классификации и до реконструкции перекрывающихся кластеров.

Ключевые слова: LXe калориметр, классификация, кластер, метод BDT.

\end{document}